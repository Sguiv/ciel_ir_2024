%%%%%%%%%%%%%%%%%%%%%%%%%%%%%%%%%%%%%%%%%
% Large Colored Title Article
% LaTeX Template
% Version 1.1 (25/11/12)
%
% This template has been downloaded from:
% http://www.LaTeXTemplates.com
%
% Original author:
% Frits Wenneker (http://www.howtotex.com)
%
% License:
% CC BY-NC-SA 3.0 (http://creativecommons.org/licenses/by-nc-sa/3.0/)
%
%%%%%%%%%%%%%%%%%%%%%%%%%%%%%%%%%%%%%%%%%

%----------------------------------------------------------------------------------------
%	PACKAGES AND OTHER DOCUMENT CONFIGURATIONS
%----------------------------------------------------------------------------------------
\documentclass[DIV=calc,paper=a4,fontsize=11pt,twocolumn,halfparskip,parindent]{scrartcl} % A4 paper and 11pt font size
\usepackage[utf8]{inputenc}  
\usepackage[francais]{babel} % French language/hyphenation
\usepackage[protrusion=true,expansion=true]{microtype} % Better typography
\usepackage{amsmath,amsfonts,amsthm} % Math packages
\usepackage[svgnames]{xcolor} % Enabling colors by their 'svgnames'
\usepackage[hang,small,labelfont=bf,up,textfont=it,up]{caption} % Custom captions under/above floats in tables or figures
\usepackage{booktabs} % Horizontal rules in tables
\usepackage{fix-cm} % Custom font sizes - used for the initial letter in the document

\usepackage{sectsty} % Enables custom section titles
\allsectionsfont{\usefont{OT1}{phv}{b}{n} \color{DarkRed}} % Change the font of all section commands

\usepackage{fancyhdr} % Needed to define custom headers/footers
\pagestyle{fancy} % Enables the custom headers/footers
\usepackage{lastpage} % Used to determine the number of pages in the document (for "Page X of Total")
\usepackage{array}
\usepackage{graphicx}
\usepackage{indentfirst} % Indentation for all paragraphs (first included)
\usepackage{bold-extra}
\usepackage{listingsutf8}
\lstset{language=C}          % Set your language
\lstset{showspaces=false,showstringspaces=false,showtabs=false}
\lstset{inputencoding=utf8/latin1}
\lstset{literate=
  {á}{{\'a}}1 {é}{{\'e}}1 {í}{{\'i}}1 {ó}{{\'o}}1 {ú}{{\'u}}1
  {Á}{{\'A}}1 {É}{{\'E}}1 {Í}{{\'I}}1 {Ó}{{\'O}}1 {Ú}{{\'U}}1
  {à}{{\`a}}1 {è}{{\`e}}1 {ì}{{\`i}}1 {ò}{{\`o}}1 {ù}{{\`u}}1
  {À}{{\`A}}1 {È}{{\'E}}1 {Ì}{{\`I}}1 {Ò}{{\`O}}1 {Ù}{{\`U}}1
  {ä}{{\"a}}1 {ë}{{\"e}}1 {ï}{{\"i}}1 {ö}{{\"o}}1 {ü}{{\"u}}1
  {Ä}{{\"A}}1 {Ë}{{\"E}}1 {Ï}{{\"I}}1 {Ö}{{\"O}}1 {Ü}{{\"U}}1
  {â}{{\^a}}1 {ê}{{\^e}}1 {î}{{\^i}}1 {ô}{{\^o}}1 {û}{{\^u}}1
  {Â}{{\^A}}1 {Ê}{{\^E}}1 {Î}{{\^I}}1 {Ô}{{\^O}}1 {Û}{{\^U}}1
  {Ã}{{\~A}}1 {ã}{{\~a}}1 {Õ}{{\~O}}1 {õ}{{\~o}}1
  {œ}{{\oe}}1 {Œ}{{\OE}}1 {æ}{{\ae}}1 {Æ}{{\AE}}1 {ß}{{\ss}}1
  {ű}{{\H{u}}}1 {Ű}{{\H{U}}}1 {ő}{{\H{o}}}1 {Ő}{{\H{O}}}1
  {ç}{{\c c}}1 {Ç}{{\c C}}1 {ø}{{\o}}1 {å}{{\r a}}1 {Å}{{\r A}}1
  {€}{{\euro}}1 {£}{{\pounds}}1 {«}{{\guillemotleft}}1
  {»}{{\guillemotright}}1 {ñ}{{\~n}}1 {Ñ}{{\~N}}1 {¿}{{?`}}1
}

% Tikz
\usepackage{tikz}
\usetikzlibrary{shapes,arrows,arrows.meta}
% Define block styles
\tikzstyle{decision}=[diamond,aspect=2,draw,fill=red!15,text width=12ex,text centered,node distance=12ex,inner sep=0pt]
%\tikzstyle{block}=[rectangle,draw,fill=blue!20,text width=5em,text centered,rounded corners,minimum height=4em]
\tikzstyle{block}=[rectangle,draw,fill=blue!20,node distance=15ex,text centered,rounded corners,text width=15ex,minimum height=5ex]
\tikzstyle{block2}=[rectangle,draw,fill=blue!70!red!10,node distance=15ex,text centered,rounded corners,text width=15ex,minimum height=5ex]
\tikzstyle{line}=[draw,-latex]
\tikzstyle{_line}=[draw,{Circle[length=5pt,fill=none]}-latex] 
%\tikzstyle{cloud}=[draw,ellipse,fill=green!20,node distance=3cm,minimum height=2em]
%\tikzstyle{cloud}=[draw,rectangle,fill=green!20,node distance=25ex,minimum height=5ex,rounded corners]
%\tikzstyle{cloud}=[draw=none,fill=none]
\tikzstyle{cloud}=[draw,circle,fill=none]

% Headers - all currently empty
\lhead{}
\chead{}
\rhead{}

% Footers
\lfoot{Loïc \textsc{Plassart}}
\cfoot{}
\rfoot{\footnotesize Page \thepage/\pageref{LastPage}} % "Page 1/2"

% Superscript
%\newcommand{\up}[1]{\textsuperscript{#1}} 

% Header en footer rules
\renewcommand{\headrulewidth}{0.0pt} % No header rule
\renewcommand{\footrulewidth}{0.4pt \color{DarkGoldenrod}} % Thin footer rule
%\renewcommand{\footrulewidth}{0.4pt \color{DarkBlue}} % Thin footer rule

%% Formatage des entêtes de noms de figures et de tableaux
%% -------------------------------------------------------
\addto\captionsfrench{\def\figurename{\textsc{Fig.}}}
\addto\captionsfrench{\def\tablename{\textsc{Tab.}}}

% Lettrine for first paragraph
\usepackage{lettrine} % Package to accentuate the first letter of the text
\newcommand{\initial}[1]{ % Defines the command and style for the first letter
\lettrine[lines=3,lhang=0.3,nindent=0em]{
\color{DarkGoldenrod}
%\color{DarkBlue}
      {\textsf{#1}}}{}}

% Space between paragraphs
\setlength{\parskip}{0.3em}

%----------------------------------------------------------------------------------------
%	TITLE SECTION
%----------------------------------------------------------------------------------------
\usepackage{titling} % Allows custom title configuration
\newcommand{\HorRule}{\color{DarkGoldenrod} \rule{\linewidth}{1pt}} % Defines the gold horizontal rule around the title
%\newcommand{\HorRule}{\color{DarkBlue} \rule{\linewidth}{1pt}} % Defines the gold horizontal rule around the title
\pretitle{\vspace{-30pt} \begin{flushleft} \HorRule \fontsize{40}{40} \usefont{OT1}{pag}{b}{n} \color{DarkRed} \selectfont} % Title font configuration
\title{Structures itératives} % Your article title
\posttitle{\par\end{flushleft}\vskip 0.5em} % Whitespace under the title
\preauthor{\begin{flushleft}\large \lineskip 0.5em \usefont{OT1}{phv}{b}{n} \color{DarkRed}} % Author font configuration
\author{Fondements du langage C} % Your name
\postauthor{\footnotesize \usefont{OT1}{phv}{m}{sl} \color{Black} % Configuration for the institution name University of California % Your institution
\par\end{flushleft}\HorRule} % Horizontal rule after the title
\date{} % Add a date here if you would like one to appear underneath the title block

%----------------------------------------------------------------------------------------
\begin{document}

\maketitle % Print the title
\thispagestyle{fancy} % Enabling the custom headers/footers for the first page 

%----------------------------------------------------------------------------------------
%	ABSTRACT
%----------------------------------------------------------------------------------------
% The first character should be within \initial{}
%\initial{H}\textbf{ere is some sample text to show the initial in the introductory paragraph of this template article. 
%The color and lineheight of the initial can be modified in the preamble of this document.}
\initial{L}\textbf{e langage C présente plusieurs types de structures itératives, c'est-à-dire, des structures de contrôle permettant de réaliser des traîtements
  cycliques en boucle. Il s'agit alors d'exécuter plusieurs fois une portion de code, généralement jusqu'à ce qu'une condition soit fausse. Il convient donc d'être
  rigoureux dans l'écriture de la condition afin de s'assurer qu'elle ne restera pas toujours vraie et que l'exécution du programme n'entrera pas dans une boucle
  infinie. Tout comme les structures conditionnelles, les structures itératives utilisent les opérateurs relationnels et logiques afin d'exprimer la condition à évaluer.}

%----------------------------------------------------------------------------------------
%	ARTICLE CONTENTS
%----------------------------------------------------------------------------------------
\section*{Boucle \texttt{while}}
La boucle \texttt{while} existe dans la plupart des langages de programmation. Elle consiste à tester une condition et à exécuter de manière itérative un bloc
d'instructions tant que cette condition reste vraie. Sa syntaxe générale est la suivante :
\begin{lstlisting}[frame=single]
while (/* Condition à évaluer */)
{
  /* Suite d'intructions à traiter */
  /* en boucle tant que la         */
  /* la condition est vraie.       */
}
\end{lstlisting}

Le bloc d'instructions à exécuter est borné par des accolades. Néanmoins, s'il s'agit de n'exécuter qu'une seule instruction, les accolades sont facultatives.

La figure \ref{while} montre la boucle \texttt{while} sous la forme d'un algorigramme. L'évaluation de la condition est effectuée préalablement à l'exécution du bloc
d'instructions.
\begin{figure}[!h]
\centering
\begin{tikzpicture}[node distance=8ex,auto]
  \node[cloud](start){};
  \node[decision,below of=start](condition){Condition vraie ?};
  \node[block2,below of=condition](execution){Bloc d'instruc-\\tions};
  \node[cloud,below of=execution,right of=execution,xshift=7ex](stop){}; 
  \path[line](start)--(condition);
  \path[line](condition)--node{Oui}(execution);
  \path[line,rounded corners](execution.south)--([yshift=-3ex]execution.south)--++(-15ex,0)|-([yshift=3ex]condition.north)--(condition.north);
  \path[_line,rounded corners](condition.east)-|node{Non}(stop);
\end{tikzpicture}
\caption{Déroulement d'une structure \texttt{while}}\label{while}
\end{figure}

Le code ci-dessous montre un exemple d'utilisation d'une boucle \texttt{while} :
\begin{lstlisting}[frame=single]
int compteur = 0;
  
while (compteur < 10)
{
  printf("Compteur : %d\n", compteur);
  compteur++;
}
\end{lstlisting}

Il est important de noter que si l'incrémentation de la variable \texttt{compteur} n'est pas effectuée, la condition de la boucle \texttt{while} qui en dépend ne pourra
jamais être fausse et le programme entrera dans une boucle infinie. 

%------------------------------------------------
\section*{Boucle \texttt{do...while}}
La boucle \texttt{do..while} se distingue de la boucle \texttt{while} par le fait que la condition est évaluée au terme du traitement du bloc d'instructions. Ainsi, et
même si la condition est initialement fausse, le bloc d'instructions est au minimum exécuté une fois. La syntaxe générale de la boucle \texttt{do...while} est la suivante :
\begin{lstlisting}[frame=single]
do
{
  /* Suite d'intructions à traiter */
  /* en boucle tant que la         */
  /* la condition est vraie.       */
} while (/* Condition à évaluer */);
\end{lstlisting}

La figure \ref{dowhile} montre la boucle \texttt{do...while} sous la forme d'un algorigramme. L'évaluation de la condition est effectuée après l'exécution du bloc
d'instructions.
\begin{figure}[!h]
\centering
\begin{tikzpicture}[node distance=8ex,auto]
  \node[cloud](start){};
  \node[block2,below of=start](execution){Bloc d'instruc-\\tions};
  \node[decision,below of=execution](condition){Condition vraie ?};
  \node[cloud,below of=condition,right of=condition,xshift=7ex](stop){}; 
  \path[line](start)--(execution);
  \path[line](execution)--(condition);
  \path[line,rounded corners](condition.south)--([yshift=-3ex]condition.south)--++(-15ex,0)|-node{Oui}([yshift=3ex]execution.north)--(execution.north);
  \path[_line,rounded corners](condition.east)-|node{Non}(stop);
\end{tikzpicture}
\caption{Déroulement d'une structure \texttt{do...while}}\label{dowhile}
\end{figure}

Il existe également un risque de boucle infinie si la condition reste systématiquement vraie. 

Le code ci-dessous montre un exemple d'utilisation d'une boucle \texttt{while} :
\begin{lstlisting}[frame=single]
int compteur = 0;
  
do
{
  printf("Compteur : %d\n", compteur);
  compteur++;
} while (compteur < 10);
\end{lstlisting}

%------------------------------------------------
\section*{Boucle \texttt{for}}
La boucle \texttt{for} est une structure très couramment utilisée dans les programmes écrits en langage C. Elle est généralement mise en \oe uvre lorsqu'il est nécessaire
de répéter un bloc d'instruction durant un nombre de fois préalablement déterminé. Sa syntaxe générale prend en compte trois champs distincts séparés par le caractère
\texttt{';'}. Elle est la suivante :\begin{lstlisting}[frame=single]
for (/* Initialisation*/;
     /* Condition */;
     /* Itération de contrôle */)
{
  /* Suite d'intructions à traiter */
  /* en boucle tant que la         */
  /* la condition est vraie.       */
}
\end{lstlisting}

L'initialisation consiste à fixer la valeur initiale d'une variable utilisée dans le contrôle de la boucle \texttt{for}. Le second champ concerne l'expression de la condition
à prendre en compte. Le troisième champ s'applique à la mise à jour de la variable de contrôle de boucle. Cette mise à jour se décline le plus souvent sous la forme d'une
incrémentation ou d'une décrémentation.

Le code ci-dessous présente un exemple classique d'implémentation d'une boucle \texttt{for} :
\begin{lstlisting}[frame=single]
int i;
  
for (i = 0; i <= 10; i++)
  printf("Valeur : %d\n", i);  
\end{lstlisting}

D'un point de vue algorithmique, la boucle \texttt{for} se comporte de façon similaire à la boucle \texttt{while}.

%------------------------------------------------
\section*{Imbrication de boucles}
Dans le cadre de traîtements particuliers, il est parfois nécessaire d'imbriquer plusieurs boucles \texttt{for}. Le code suivant montre un cas d'application :
\begin{lstlisting}[frame=single]
int i, j;
  
for (i = 0; i < 10; i++)
  for (j = 0; j < 10; j++)
    printf("Addition : %d\n", i + j);  
\end{lstlisting}
%----------------------------------------------------------------------------------------
\end{document}
