%%%%%%%%%%%%%%%%%%%%%%%%%%%%%%%%%%%%%%%%%
% Large Colored Title Article
% LaTeX Template
% Version 1.1 (25/11/12)
%
% This template has been downloaded from:
% http://www.LaTeXTemplates.com
%
% Original author:
% Frits Wenneker (http://www.howtotex.com)
%
% License:
% CC BY-NC-SA 3.0 (http://creativecommons.org/licenses/by-nc-sa/3.0/)
%
%%%%%%%%%%%%%%%%%%%%%%%%%%%%%%%%%%%%%%%%%

%----------------------------------------------------------------------------------------
%	PACKAGES AND OTHER DOCUMENT CONFIGURATIONS
%----------------------------------------------------------------------------------------
\documentclass[DIV=calc,paper=a4,fontsize=11pt,twocolumn,halfparskip,parindent]{scrartcl} % A4 paper and 11pt font size
\usepackage[utf8]{inputenc}  
\usepackage[francais]{babel} % French language/hyphenation
\usepackage[protrusion=true,expansion=true]{microtype} % Better typography
\usepackage{amsmath,amsfonts,amsthm} % Math packages
\usepackage[svgnames]{xcolor} % Enabling colors by their 'svgnames'
\usepackage[hang,small,labelfont=bf,up,textfont=it,up]{caption} % Custom captions under/above floats in tables or figures
\usepackage{booktabs} % Horizontal rules in tables
\usepackage{fix-cm} % Custom font sizes - used for the initial letter in the document

\usepackage{sectsty} % Enables custom section titles
\allsectionsfont{\usefont{OT1}{phv}{b}{n} \color{DarkRed}} % Change the font of all section commands

\usepackage{fancyhdr} % Needed to define custom headers/footers
\pagestyle{fancy} % Enables the custom headers/footers
\usepackage{lastpage} % Used to determine the number of pages in the document (for "Page X of Total")
\usepackage{array}
\usepackage{graphicx}
\usepackage{indentfirst} % Indentation for all paragraphs (first included)
\usepackage{bold-extra}
\usepackage{listingsutf8}
\lstset{language=C}          % Set your language
\lstset{showspaces=false,showstringspaces=false,showtabs=false}
\lstset{inputencoding=utf8/latin1}
\lstset{literate=
  {á}{{\'a}}1 {é}{{\'e}}1 {í}{{\'i}}1 {ó}{{\'o}}1 {ú}{{\'u}}1
  {Á}{{\'A}}1 {É}{{\'E}}1 {Í}{{\'I}}1 {Ó}{{\'O}}1 {Ú}{{\'U}}1
  {à}{{\`a}}1 {è}{{\`e}}1 {ì}{{\`i}}1 {ò}{{\`o}}1 {ù}{{\`u}}1
  {À}{{\`A}}1 {È}{{\'E}}1 {Ì}{{\`I}}1 {Ò}{{\`O}}1 {Ù}{{\`U}}1
  {ä}{{\"a}}1 {ë}{{\"e}}1 {ï}{{\"i}}1 {ö}{{\"o}}1 {ü}{{\"u}}1
  {Ä}{{\"A}}1 {Ë}{{\"E}}1 {Ï}{{\"I}}1 {Ö}{{\"O}}1 {Ü}{{\"U}}1
  {â}{{\^a}}1 {ê}{{\^e}}1 {î}{{\^i}}1 {ô}{{\^o}}1 {û}{{\^u}}1
  {Â}{{\^A}}1 {Ê}{{\^E}}1 {Î}{{\^I}}1 {Ô}{{\^O}}1 {Û}{{\^U}}1
  {Ã}{{\~A}}1 {ã}{{\~a}}1 {Õ}{{\~O}}1 {õ}{{\~o}}1
  {œ}{{\oe}}1 {Œ}{{\OE}}1 {æ}{{\ae}}1 {Æ}{{\AE}}1 {ß}{{\ss}}1
  {ű}{{\H{u}}}1 {Ű}{{\H{U}}}1 {ő}{{\H{o}}}1 {Ő}{{\H{O}}}1
  {ç}{{\c c}}1 {Ç}{{\c C}}1 {ø}{{\o}}1 {å}{{\r a}}1 {Å}{{\r A}}1
  {€}{{\euro}}1 {£}{{\pounds}}1 {«}{{\guillemotleft}}1
  {»}{{\guillemotright}}1 {ñ}{{\~n}}1 {Ñ}{{\~N}}1 {¿}{{?`}}1
}

% Tikz
\usepackage{tikz}
\usetikzlibrary{shapes,arrows,arrows.meta}
% Define block styles
\tikzstyle{decision}=[diamond,aspect=2,draw,fill=red!15,text width=12ex,text centered,node distance=12ex,inner sep=0pt]
%\tikzstyle{block}=[rectangle,draw,fill=blue!20,text width=5em,text centered,rounded corners,minimum height=4em]
\tikzstyle{block}=[rectangle,draw,fill=blue!20,node distance=15ex,text centered,rounded corners,text width=15ex,minimum height=5ex]
\tikzstyle{block2}=[rectangle,draw,fill=blue!70!red!10,node distance=15ex,text centered,rounded corners,text width=15ex,minimum height=5ex]
\tikzstyle{line}=[draw,-latex]
\tikzstyle{_line}=[draw,{Circle[length=5pt,fill=none]}-latex] 
%\tikzstyle{cloud}=[draw,ellipse,fill=green!20,node distance=3cm,minimum height=2em]
%\tikzstyle{cloud}=[draw,rectangle,fill=green!20,node distance=25ex,minimum height=5ex,rounded corners]
%\tikzstyle{cloud}=[draw=none,fill=none]
\tikzstyle{cloud}=[draw,circle,fill=none]

% Headers - all currently empty
\lhead{}
\chead{}
\rhead{}

% Footers
\lfoot{Loïc \textsc{Plassart}}
\cfoot{}
\rfoot{\footnotesize Page \thepage/\pageref{LastPage}} % "Page 1/2"

% Superscript
%\newcommand{\up}[1]{\textsuperscript{#1}} 

% Header en footer rules
\renewcommand{\headrulewidth}{0.0pt} % No header rule
\renewcommand{\footrulewidth}{0.4pt \color{DarkGoldenrod}} % Thin footer rule
%\renewcommand{\footrulewidth}{0.4pt \color{DarkBlue}} % Thin footer rule

%% Formatage des entêtes de noms de figures et de tableaux
%% -------------------------------------------------------
\addto\captionsfrench{\def\figurename{\textsc{Fig.}}}
\addto\captionsfrench{\def\tablename{\textsc{Tab.}}}

% Lettrine for first paragraph
\usepackage{lettrine} % Package to accentuate the first letter of the text
\newcommand{\initial}[1]{ % Defines the command and style for the first letter
\lettrine[lines=3,lhang=0.3,nindent=0em]{
\color{DarkGoldenrod}
%\color{DarkBlue}
      {\textsf{#1}}}{}}

% Space between paragraphs
\setlength{\parskip}{0.3em}

%----------------------------------------------------------------------------------------
%	TITLE SECTION
%----------------------------------------------------------------------------------------
\usepackage{titling} % Allows custom title configuration
\newcommand{\HorRule}{\color{DarkGoldenrod} \rule{\linewidth}{1pt}} % Defines the gold horizontal rule around the title
%\newcommand{\HorRule}{\color{DarkBlue} \rule{\linewidth}{1pt}} % Defines the gold horizontal rule around the title
\pretitle{\vspace{-30pt} \begin{flushleft} \HorRule \fontsize{40}{40} \usefont{OT1}{pag}{b}{n} \color{DarkRed} \selectfont} % Title font configuration
\title{Structures conditionnelles} % Your article title
\posttitle{\par\end{flushleft}\vskip 0.5em} % Whitespace under the title
\preauthor{\begin{flushleft}\large \lineskip 0.5em \usefont{OT1}{phv}{b}{n} \color{DarkRed}} % Author font configuration
\author{Fondements du langage C} % Your name
\postauthor{\footnotesize \usefont{OT1}{phv}{m}{sl} \color{Black} % Configuration for the institution name University of California % Your institution
\par\end{flushleft}\HorRule} % Horizontal rule after the title
\date{} % Add a date here if you would like one to appear underneath the title block

%----------------------------------------------------------------------------------------
\begin{document}

\maketitle % Print the title
\thispagestyle{fancy} % Enabling the custom headers/footers for the first page 

%----------------------------------------------------------------------------------------
%	ABSTRACT
%----------------------------------------------------------------------------------------
% The first character should be within \initial{}
%\initial{H}\textbf{ere is some sample text to show the initial in the introductory paragraph of this template article. 
%The color and lineheight of the initial can be modified in the preamble of this document.}
\initial{L}\textbf{es structures conditionnelles sont très utilisées en programmation pour conditionner l'exécution d'un programme. Du point de vue algorithmique,
  il s'agit de mécanismes qui permettent d'exprimer une décision et de sélectionner des suites d'instructions à traîter en fonction d’une condition (prédicat) qui
  peut être vraie ou fausse. Une condition est ainsi une expression dont la valeur va déterminer les actions réalisées dans la suite de l'exécution d'un programme.}

%----------------------------------------------------------------------------------------
%	ARTICLE CONTENTS
%----------------------------------------------------------------------------------------
\section*{Opérateurs relationnels et logiques}
Dès lors que le principe d'une structure conditionnelle repose sur l'évaluation de conditions, il est nécessaire de faire appel à des opérateurs permettant d'établir
des expressions relationnelles et logiques. Par défaut, une telle expression est associée au type \texttt{int} et prend pour valeur 0 si elle est fausse et 1 si elle
est vraie. Par ailleurs, toute expression non nulle est logiquement évaluée comme étant vraie.

Les opérateurs relationnels et logiques sont les listés dans le tableau~\ref{operateurs}:
\begin{table}[!h]
\centering
\begin{tabular}{ll}
\toprule
Opérateur & Signification\\
\midrule\midrule
$>$ & Strictement supérieur\\
$<$ & Strictement inférieur\\
$>=$ & Supérieur ou égal\\
$<=$ & Inférieur ou égal\\
$==$ & \'Egal\\
$!=$ & Non égal\\
\&\& & ET logique\\
\textbar\textbar & OU logique\\
$!$ & NON logique\\
\bottomrule
\end{tabular}
\caption{Liste des opérateurs relationnels et logiques}\label{operateurs}
\end{table}

Il est alors possible de combiner plusieurs expressions relationnelles au moyen d'opérateurs logiques tel que le montre l'exemple ci-dessous :
\begin{lstlisting}[frame=single]
(age >= 8) && (age <= 12)
\end{lstlisting}

Dans cet exemple, la condition à évaluer n'est vraie que si les deux expressions relationnelles sont vraies.

L'utilisation de l'opérateur NON logique permet d'évaluer l'inverse de l'expression :
\begin{lstlisting}[frame=single]
!(age >= 18)
\end{lstlisting}

L'exemple ci-dessus exprime une condition qui est vraie si la valeur de la variable \texttt{age} n'est pas supérieure ou égale à 18, c'est-à-dire qu'elle est
inférieure à 18. 

%------------------------------------------------
\section*{Structure \texttt{\textbf{if...else}}}
La structure conditionnelle la plus courante et la plus basique est \texttt{if...else}. Son principe repose sur l’évaluation booléenne d’une condition. Sa syntaxe
générale est la suivante :
\begin{lstlisting}[frame=single]
if (/* Condition à évaluer */)
{
  /* Suite d'intructions à       */
  /* traiter si la condition est */
  /* vraie.                      */
}
else
{
  /* Autre suite d'intructions à  */
  /* traiter si la condition est  */
  /* si la condition est fausse.  */
}
\end{lstlisting}

La figure~\ref{ifelse} présente sous forme d'un algorigramme le déroulement d'une structure \texttt{if...else}. Elle schématise le fait que l'exécution du
programme ne peut s'appliquer qu'à l'un des deux blocs d'instructions sélectionné en fonction du résultat de l'évaluation de la condition.
\begin{figure}[!h]
\centering
\begin{tikzpicture}[node distance=8ex,auto]
  \node[cloud](start){};
  \node[decision,below of=start](condition){Condition vraie ?};
  \node[block2,below of=condition](execution-1){Bloc d'instruc-\\tions \texttt{if}};
  \node[block2,right of=execution-1,node distance=20ex](execution-2){Bloc d'instructions \texttt{else}};
  \node[cloud,below of=execution-1](stop){}; 
  \path[line](start)--(condition);
  \path[line](condition)--node{Oui}(execution-1);
  \path[_line,rounded corners](condition)-|node{Non}(execution-2);
  \path[line](execution-1)--(stop);
  \path[line,rounded corners](execution-2)|-node{}(stop);
\end{tikzpicture}
\caption{Déroulement d'une structure \texttt{if...else}}\label{ifelse}
\end{figure}

Lorsqu’il est nécessaire d’exécuter plusieurs instructions selon que la condition est vérifiée ou non, il faut créer un bloc de code, c’est-à-dire un ensemble
d’instructions borné par des accolades (\{ et \}). Cependant et même lorsqu'une seule instruction doit être traitée, il est conseillé d’utiliser systématiquement
les accolades. Le code ci-dessous présente un exemple d’utilisation de la structure \texttt{if...else} :
\begin{lstlisting}[frame=single]
if (age >= 18)
{
  printf("Vous êtes majeur !");
}
else
{
  printf("Vous êtes mineur !");
}
\end{lstlisting}

Le cas \texttt{else} n’est pas directement associé à condition. Il correspond en effet au traitement par défaut, c’est-à-dire qu’il est pris en compte uniquement lorsque
la condition booléenne exprimée au niveau du \texttt{if} n’est pas vérifiée. 

L'exemple ci-dessous montre l'utilisation du \texttt{if} considérant deux expressions relationnelles liées par l'opérateur ET logique :
\begin{lstlisting}[frame=single]
if ((speed >= 150) && (speed < 200))
{
  printf("Rapide !");
}
\end{lstlisting}

Un autre exemple présenté ci-dessous permet d'exclure les cas limites :
\begin{lstlisting}[frame=single]
if ((speed == 0) || (speed >= 300))
{
  printf("Hors limites !");
}
\end{lstlisting}

Dans cet exemple, c'est l'opérateur OU logique qui est utilisé.

Il est également possible d’imbriquer les cas d’évaluation d’une condition en utilisant le test \texttt{else if}. Il s'agit alors d'une structure \texttt{if...else}
généralisée permettant de prendre en compte un plus grand nombre de cas. Le code suivant montre un exemple d’implémentation :
\begin{lstlisting}[frame=single]
if (vitesse >= 200)
{
  printf("Très rapide !");
}
else if (vitesse >= 150)
{
  printf("Rapide !");
}
else if (vitesse >= 100)
{
  printf("Moyen !");
}
else
{
  printf("Lent !");
}
\end{lstlisting}

L'algorigramme présenté par la figure~\ref{ifelseif} montre le déroulement d'une structure de type \texttt{if...else if...else} basée sur un ensemble de trois
évaluations de condition. Ces différentes évaluations sont successivement considérées jusqu'à ce que l'une d'entre-elles soit vérifiée. Le cas échéant, le bloc
d'instructions correspondant est traité. Dans le cas contraire, c'est le bloc d'instructions par défaut et associé au \texttt{else} qui est exécuté.
\begin{figure*}[!h]
\centering
\begin{tikzpicture}[node distance=8ex,auto]
  \node[cloud](start){};
  \node[decision,below of=start](condition-1){Condition 1 vraie ?};
  \node[decision,right of=condition-1,below of=condition-1,node distance=15ex,xshift=5ex](condition-2){Condition 2 vraie ?};
  \node[decision,right of=condition-2,below of=condition-2,node distance=15ex,xshift=5ex](condition-3){Condition 3 vraie ?};
  \node[block2,below of=condition-3](execution-3){Bloc d'instruc-\\tions \texttt{else if}};
  \node[block2,left of=execution-3,xshift=-5ex](execution-2){Bloc d'instruc-\\tions \texttt{else if}};
  \node[block2,left of=execution-2,xshift=-5ex](execution-1){Bloc d'instruc-\\tions \texttt{if}};
  \node[block2,right of=execution-3,xshift=5ex](execution-4){Bloc d'instruc-\\tions \texttt{else}};
  \node[cloud,below of=execution-1](stop){}; 
  \path[line](start)--(condition-1);
  \path[line](condition-1)--node{Oui}(execution-1);
  \path[line](condition-2)--node{Oui}(execution-2);
  \path[line](condition-3)--node{Oui}(execution-3);
  \path[_line,rounded corners](condition-3)-|node{Non}(execution-4);
  \path[_line,rounded corners](condition-2)-|node{Non}(condition-3);
  \path[_line,rounded corners](condition-1)-|node{Non}(condition-2);
  \path[line,rounded corners](execution-1)--node{}(stop);
  \path[line,rounded corners](execution-2)|-node{}(stop);
  \path[line,rounded corners](execution-3)|-node{}(stop);
  \path[line,rounded corners](execution-4)|-node{}(stop);
\end{tikzpicture}
\caption{Déroulement d'une structure \texttt{if...else if...else}}\label{ifelseif}
\end{figure*}

%------------------------------------------------
\section*{Structure \texttt\textbf{switch...case}}
La structure \texttt{switch...case} s'applique lorsqu'un choix multiple doit être effectué. En termes d'usage et de représentation algorithmique, elle s'apparente à
une structure \texttt{if...else} généralisée. L'instruction \texttt{switch} permet de faire plusieurs tests de valeurs sur le contenu d'une même variable. Ce branchement
conditionnel simplifie beaucoup le test de plusieurs valeurs d'une variable en évitant l’utilisation d'une structure de type \texttt{if...else if...else}. Sa syntaxe
générale est la suivante : 
\begin{lstlisting}[frame=single]
switch (/* Variable à évaluer */)
{
  case /* Valeur 1 */ :
  /* Suite d'instructions */
  /* à traiter            */
  break;

  case /* Valeur 2 */ :
  /* Suite d'instructions */
  /* à traiter            */
  break;

  case /* Valeur 3 */ :
  /* Suite d'instructions */
  /* à traiter            */
  break;

  default :
  /* Suite d'instructions */
  /* à traiter            */
}	
\end{lstlisting}

Chaque cas correspond à une confrontation de la valeur de variable à évaluer avec une valeur constante. Lorsque l'égalité est vérifiée, le bloc d'instructions
associé au cas est exécuté.

L'instruction \texttt{break} provoque la sortie du bloc \texttt{switch}. Si elle n'est pas précisée, le cas suivant est également évalué. Cette situation n'a généralement
aucun intérêt et il convient le plus souvent d'utiliser l'instruction \texttt{break} afin d'éviter d'exécuter des tests inutiles.

Le cas \texttt{default} est facultatif comme peut l'être celui du \texttt{else} dans une structure \texttt{if...else}. Il est exécuté seulement si la valeur de la
variable à évaluer ne correspond à aucun des cas précémment évalués. 
\begin{lstlisting}[frame=single]
switch (commande)
{
  case 'M' :
  printf("Démarrage !\n");
  break;

  case 'A' :
  printf("Arrêt !\n");
  break;

  default :
  printf("Ordre non valide !\n");
}	
\end{lstlisting}

%----------------------------------------------------------------------------------------
%	REFERENCE LIST
%----------------------------------------------------------------------------------------
%\begin{thebibliography}{99} % Bibliography - this is intentionally simple in this template

%\bibitem[Figueredo and Wolf, 2009]{Figueredo:2009dg}
%Figueredo, A.~J. and Wolf, P. S.~A. (2009).
%\newblock Assortative pairing and life history strategy - a cross-cultural study.
%\newblock {\em Human Nature}, 20:317--330.
 
%\end{thebibliography}

%----------------------------------------------------------------------------------------
\end{document}
